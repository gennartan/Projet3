\documentclass{beamer}
 
 \usepackage[utf8x]{inputenc}
\usepackage[T1]{fontenc}
\usepackage[french]{babel}

\usepackage{lmodern, graphicx, tikz, pgfplots}
\usepackage{graphicx,amssymb,amstext,amsmath}
\usepackage{fancyhdr, url, ifthen, multirow}
\usepackage{subfig, float, caption}
\usepackage{array}
\usepackage{color, colortbl}
\usepackage{hyperref, siunitx}



\begin{document}

\title[Présentation]{Présentation Laboratoire}

\subtitle[\ldots]{Polarisation des ondes}
\author{Groupe 11.64}
\institute[UCL]{Ecole polytechnique de Louvain}
\date{\today}
\maketitle

\begin{frame}
\frametitle{Trouver l'indice de réfraction de la lucite}
\framesubtitle{Salut cite}

\begin{tabular}{|c|c|c|}
	\hline
		Angle d'incidence & Angle de réflexion ($\theta_a$) & Angle de réfraction ($\theta_b$) \\
	\hline
		0 & 0 & 0 \\
		10 & 10 & 7 \\
		20 & 20 & 14 \\
		30 & 30 & 20 \\
		40 & 40 & 26 \\
		50 & 50 & 31.5 \\
		60 & 60 & 36 \\
		70 & 70 & 39 \\
		80 & 80 & 41.5 \\
	\hline
	\end{tabular}
	
	Par la relation $ \frac{n_b}{n_a} = \frac{\sin (\theta_a)}{\sin (\theta_b)} $ on trouve que $n_b$\footnote{$n_b$ est la l'indice de réfraction de la lucite et $n_a$ est égal a 1} $= 1.4$ 

\end{frame} 

\begin{frame}
\frametitle{L'angle de réflexion total et l'angle de Brewster}
\framesubtitle{total excellium}
C'est l'angle a partir duquel le faisceau n'est plus du tout réfracté et est totalement réfléchi.
$$ \sin (\theta_{crit} ) = \frac{n_b}{n_a} $$
Donc $$ \theta_{crit} = asin (\frac{n_a}{n_b} ) $$
$$ \theta_{crit} = 45 $$

L'angle de \textbf{Brewster} est l'angle à partir duquel l'onde réfléchie s'annule pour une onde incidente polarisée parallèlement au plan d'incidence.

$$ \tan (\theta_p) = \frac{nb}{na} $$
$$ \theta_p = 55.6 \deg $$

Par contre pour une onde polarisée perpendiculairement au plan d'incidence, la réflectance n'est pas nulle à l'ange de \textbf{Brewster}.

\end{frame}

\begin{frame}
\frametitle{Polarisation du faisceau réfléchi en fonction de l'angle d'incidence}
\framesubtitle{le faisceau spatial}
Le laser est incliné et polarisé à 45 degré. Nous mesurons l'état de polarisation du faisceau réfléchi.
\begin{tabular}{|c|c|}
	\hline
		Angle d'incidence & Etat de polarisation\footnote{par rapport a la verticale)} \\
	\hline
		10 & 45 \\
		20 & 40 \\
		30 & 35 \\ 
		40 & 20 \\
		50 & 10 \\
		55 & 0 \\
		60 & (-)5 \\
		70 & (-)5 \\
	\hline
	\end{tabular}
	\newline On remarque qu'à l'angle de \textbf{Brewster} le faisceau est polarisé à 0 degré (vertical), et que passé cet angle, le faisceau reste polarisé proche de 0 degré.
\end{frame}

\begin{frame}
\frametitle{Et on plot}
\framesubtitle{plot}

\begin{tikzpicture}
			\begin{axis}[
				height=6cm,
				width=10cm,
				grid=major,
				legend pos=south east,
				xlabel={Angle d'incidence},
				ylabel={Etat de polarisation},
			]
			\addplot coordinates {
				(45, 10)
				(40, 20)
				(35, 30)
				(20, 40)
				(10, 50)
				(0, 55)
				(-5, 60)
				(-5, 70)
			};
			\addlegendentry{machin}
			\end{axis}
		\end{tikzpicture}
\end{frame}
\end{document}