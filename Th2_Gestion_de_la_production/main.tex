\documentclass[a4paper,french]{article}

\usepackage[T1]{fontenc}
\usepackage[utf8x]{inputenc}
\usepackage{chemist}
\usepackage{siunitx}
\usepackage{lmodern}
\usepackage{babel}
\usepackage{float}
\usepackage{fullpage}


\title{Thématique 2 : Gestion de la production}
\author{Groupe 11.64}
\date{23/09/2015\\Dernière mise à jour: \today}

\begin{document}
	\maketitle	
	\section{Synthèse de l'ammoniac}
		\begin{chemmath}
			3H_2 + N_2 \longrightarrow 2NH_3	
		\end{chemmath}
		\begin{table}[h]
			\centering
			\renewcommand{\arraystretch}{2}
			\begin{tabular}{|cl|cl|cl|}\hline
				\chemform{3H_2} & $m = \SI{250e9}{\gram}$ & \chemform{N_2} & $m = \SI{1167e9}{\gram}$ & \chemform{2NH_3} & $m = \SI{1417e9}{\gram}$ \\
				& $n = \SI{125e6}{\mol}$ & & $n = \SI{41.7e6}{\mol}$ & & $n = \SI{83.3e6}{\mol}$ \\
				& $M = \SI{2.0}{\gram\per\mol}$ & & $M = \SI{28}{\gram\per\mol}$ & & $M = \SI{17}{\gram\per\mol}$ \\\hline
			\end{tabular}
		\end{table}
	\section{Aspect thermique}
	\section{Réactions ATR I et WGS}
		ATR I :
		\begin{chemmath}
			2O_2 + 2CH_4 + H_2O \longrightarrow CH_4 + 3H_2O + CO_2
		\end{chemmath}
		\indent WGS :
		\begin{chemmath}
			2H_2O + CO_2 + CO + H_2 \longrightarrow H_2O + 2CO_2 + 2H_2
		\end{chemmath}
	\section{Bilan des données}
		\begin{table}[h]
			\begin{tabular}{ll}
				0 & Quelle quantité de \chemform{O_2} et \chemform{N_2} nécessaire? \\
				1a & Quelle quantité d'énergie libérée lors de la combustion? \\
				1a & Quelle quantité de \chemform{CH_4} et \chemform{H_2O} \\
				& (Pourquoi rentrée d'\chemform{H_2O} dans zone de combustion?) \\
				1b & Zone de reformage $\rightarrow$ 2 réactions incomplètes \\
				& $\circ$ $K_c$ des 2 réactions \\
				& $\circ$ Energie libérée ou absorbée par chaque réaction \\
			2 & Energie absorbée ou libérée par la réaction \\
			3 & Technique d'élimination du \chemform{CO_2}/\chemform{H_2O} + énergie \\
			4 & Voir séance 1 \\
			& Température dans chaque bloc
			\end{tabular}
		\end{table}
\end{document}
