\documentclass[french, a4paper, 10pt]{article}

\usepackage{babel}
\usepackage[T1]{fontenc}
\usepackage[utf8x]{inputenc}

\usepackage{fullpage}
\usepackage{float}
\usepackage{nicefrac}

\usepackage{chemist}
\usepackage{siunitx}

\DeclareSIUnit\jour{j}
\sisetup{per-mode=fraction, fraction-function=\nicefrac}
\newcommand{\dotc}[2]{\dot{#1}_{\text{\chemform{#2}}}}

\title{Thématique 2 : Gestion de la production}
\author{Groupe 12.64}
\date{\today}

\begin{document}
\maketitle
\part{Bilan de matière}
Dans cette partie nous calculerons pour toutes les étapes du procédé les bilans de matière. À base de ceux-ci nous déterminerons les débits de production d'ammoniac, d'alimentation d'air, ainsi que tous les débits intermédiares entres les unités opérationnelles. Nous étudierons aussi un cas de production d'ammoniac avec des paramètres donnés.

Nos paramètres ainsi que leurs valeurs pour le cas précis sont les suivants :
\begin{table}[h]
	\centering\renewcommand{\arraystretch}{1.1}
	\begin{tabular}{lccl}\hline
		Paramètre & Valeur & Unité & Description \\\hline
		$\dotc{m}{CH_4}$ & 800 & \si{\tonne\per\jour} & Débit massique d'alimentation de \chemform{CH_4} \\
		$\nicefrac{\text{\chemform{O_2}}}{\text{\chemform{CH_4}}}$ & 0.6 & - & Rapport $\nicefrac{\text{\chemform{O_2}}}{\text{\chemform{CH_4}}}$ à l'entrée de l'ATR \\
		$\nicefrac{\text{\chemform{H_2O}}}{\text{\chemform{CH_4}}}$& 1.5 & - & Rapport $\nicefrac{\text{\chemform{H_2O}}}{\text{\chemform{CH_4}}}$ à l'entrée de l'ATR \\
		$T_{\text{ATR}}$ & 1200 & \si{\kelvin} & Température de la zone reforming de l'ATR \\
		$p_{\text{ATR}}$ & 50   & \si{\bar} & Pression d'opération de l'ATR \\\hline
	\end{tabular}
	\caption{\label{tab:parametres}Paramètres influant le fonctionnement du procédé}
\end{table}

Pour faciliter nos calculs nous considérons le symbole $\dot{n}$ comme étant le nombre de \si{\mega\mol} (\SI{1e6}{\mol}) produit par jour.
\section{Zone de combustion}
Nous commencons par la zone de combustion où se produit la réaction chimique suivante :
	\begin{chemeqn}
		CH_4 + 2O_2 \longrightarrow CO_2 + 2H_2O
	\end{chemeqn}
On considère la réaction comme étant complète avec un excès de \chemform{CH_4}. 
\begin{table}[h]
	\centering\renewcommand{\arraystretch}{1.2}
	\begin{tabular}{|l|ccccccc|}\hline
		& \chemform{CH_4} & + & \chemform{2O_2} & $\longrightarrow$ & \chemform{CO_2} & + & \chemform{2H_2O} \\\hline
		$\dot{n}_i \,[\si{\mega\mol\per\jour}]$ & $\dotc{n}{CH_4}$ && $\dotc{n}{CO_2}$ && 0  && 0  \\
		$\dot{n}_f \,[\si{\mega\mol\per\jour}]$	& $\dotc{n}{CH_4}-\dotc{n}{CO_2}$ && 0  && $\dotc{n}{CO_2}$ && $2\dotc{n}{CO_2}$ \\\hline
	\end{tabular}
\end{table}
\section{Zone de reformage}
Deux équations à deux inconnues $\xi$ et $\gamma$:
$$K_1 = \frac{(\xi-\gamma)(3\xi+\gamma)^3}{(\dotc{n}{CH_4}+\dotc{n}{H_2O}+2\xi-\gamma)^2}\cdot\frac{p_t^2}{p_0^2}\cdot\frac{1}{(\dotc{n}{H_2O}-\xi-\gamma)(\dotc{n}{CH_4}-\gamma)}$$
$$K_2 = \frac{(\dotc{n}{CO_2}+\gamma)(3\xi+\gamma)}{(\xi-\gamma)(\dotc{n}{H_2O}-\xi-\gamma)}$$
\end{document}
