\documentclass[french, a4paper, 10pt]{article}

\usepackage{babel}
\usepackage[T1]{fontenc}
\usepackage[utf8x]{inputenc}

\usepackage{fullpage}
\usepackage{float}
\usepackage{nicefrac}

\usepackage{chemist}
\usepackage{siunitx}
\usepackage{amsmath}

\DeclareSIUnit\jour{j}
\sisetup{per-mode=fraction, fraction-function=\nicefrac}
\newcommand{\dotc}[2]{\dot{#1}_{\text{\chemform{#2}}}}

\title{Thématique 2 : Gestion de la production}
\author{Groupe 12.64}
\date{\today}

\begin{document}
\maketitle
\tableofcontents
\part{Bilan de matière}
Dans cette partie nous calculerons pour toutes les étapes du procédé les bilans de matière. À base de ceux-ci nous déterminerons les débits de production d'ammoniac, d'alimentation d'air, ainsi que tous les débits intermédiares entres les unités opérationnelles. Nous étudierons aussi un cas de production d'ammoniac avec des paramètres donnés.

Pour faciliter nos calculs nous considérons le symbole $\dot{n}$ comme étant le nombre de \si{\mega\mol} (\SI{1e6}{\mol}) produit par jour.
Les débits molaires que nous utilisons dans nos calculs dépendent de la zone dans laquelle ils sont calculés. Par exemple le $\dotc{n}{CH_4}$ de la zone de combustion n'est pas le même que celui de la zone de reformage.

\section{Calcul analytique des débits}
\subsection{Unité ATR}
\subsubsection{Zone de combustion}
Nous commencons par la zone de combustion où se produit la réaction chimique suivante :
	\begin{chemeqn}
		CH_4 + 2O_2 \longrightarrow CO_2 + 2H_2O
	\end{chemeqn}
On considère la réaction comme étant complète avec un excès de \chemform{CH_4}. 

\begin{table}[h]
	\centering\renewcommand{\arraystretch}{1.2}
	\begin{tabular}{l|ccccccc}
		& \chemform{CH_4} & + & \chemform{2O_2} & $\longrightarrow$ & \chemform{CO_2} & + & \chemform{2H_2O} \\\hline
		$\dot{n}_i$ & $\dotc{n}{CH_4}$ && $2\dotc{n}{O_2}$ && 0  && $\dotc{n}{H_2O}$  \\
		$\dot{n}_f$	& $\dotc{n}{CH_4}-\dotc{n}{O_2}$ && 0  && $\dotc{n}{O_2}$ && $\dotc{n}{H_2O}+2\dotc{n}{O_2}$ \\
	\end{tabular}
	\caption{\label{tab:combustion}Avancement de la combustion}
\end{table}

\subsubsection{Zone de reformage}
Deux équilibres chimiques ont lieu dans la zone de reformage. Ces deux réactions se déroulant simultanément, on ne peut les considérer séparement : 
\begin{chemeqn}CH_4+H_2O \rightleftharpoons 3H_2 + CO\end{chemeqn}
\begin{chemeqn}CO + H_2O \rightleftharpoons H_2 + CO_2\end{chemeqn}

\begin{table}[H]
	\centering\renewcommand{\arraystretch}{1.1}
	\begin{tabular}{l|ccccccc}
		& \chemform{CH_4} & + & \chemform{H_2O} & $\rightleftharpoons$ & \chemform{3H_2} & + & \chemform{CO} \\\hline 
		$\dot{n_0}$ & $\dotc{n}{CH_4}$ && $\dotc{n}{H_2O}$ && 0 && 0 \\
	   	$\dot{n_f}$ & $\dotc{n}{CH_4}-\xi$ && $\dotc{n}{H_2O}-\xi-\beta$ && $3\xi+\beta$ && $\xi-\beta$ \\	
	\end{tabular}
	\caption{\label{tab:reformage1}Avancement du premier équilibre chimique}
\end{table}

\begin{table}[H]
	\centering\renewcommand{\arraystretch}{1.1}
	\begin{tabular}{l|ccccccc}
		& \chemform{CO} & + & \chemform{H_2O} & $\rightleftharpoons$ & \chemform{H_2} & + & \chemform{CO_2} \\\hline 
		$\dot{n_0}$ & 0 && $\dotc{n}{H_2O}$ && 0 && $\dotc{n}{CO_2}$ \\
	   	$\dot{n_f}$ & $\xi-\beta$ && $\dotc{n}{H_2O}-\xi-\beta$ && $3\xi+\beta$ && $\dotc{n}{CO_2}+\beta$ \\	
	\end{tabular}
	\caption{\label{tab:reformage2}Avancement du deuxième équilibre chimique}
\end{table}

On déduit ensuite pour chaque équilibre l'expression de la constante d'équilibre :
\begin{align}
K_1 &= \frac{(\xi-\beta)(3\xi+\beta)^3}{(\dotc{n}{H_2O}-\xi-\beta)(\dotc{n}{CH_4}-\beta)}\cdot\frac{p_t^2}{(\dotc{n}{CH_4}+\dotc{n}{H_2O}+2\xi-\beta)^2}\\
K_2 &= \frac{(\dotc{n}{CO_2}+\beta)(3\xi+\beta)}{(\xi-\beta)(\dotc{n}{H_2O}-\xi-\beta)}
\end{align}
À partir de formules connues et de la température dans l'ATR donnée en paramètre nous pouvons déterminer les valeurss de $K_1$ et $K_2$ : 
$$ \begin{cases}K_1 &= 10^{\left(\frac{-11650}{T}+13.076\right)} \\
				K_2 &= 10^{\left(\frac{1910}{T}-1.764\right)} \end{cases}$$
Nous avons alors un système de deux équations à deux inconnues ($\xi$ et $\beta$). Nous résolvons ce système à l'aide du logiciel Matlab avec la fonction suivante :

Au final, nous obtenons à la sortie de la zone de reformage les débits suivants :
\begin{align*}
	\dotc{n'}{CH_4} &= \dotc{n}{CH_4}-\xi\\
	\dotc{n'}{H_2O} &= \dotc{n}{H_2O}-\xi-\beta\\
	\dotc{n'}{CO_2} &= \dotc{n}{CO_2}+\beta\\
	\dotc{n'}{CO}   &= \xi-\beta\\
	\dotc{n'}{H_2}  &= 3\xi+\beta
\end{align*}

\subsection{Water Gas Shift (WGS)}
Dans la zone de reformage se déroule déjà une réaction Water Gas Shift. Cette réaction se déroule de manière incomplète mais dans cette unité opérationnelle nous la considérons comme complète : 
\begin{chemeqn}CO + H_2O \longrightarrow CO_2 + H_2\end{chemeqn}

\begin{table}[h]
	\centering\renewcommand{\arraystretch}{1.2}
	\begin{tabular}{l|ccccccc}
		& \chemform{CO} & + & \chemform{H_2O} & $\longrightarrow$ & \chemform{CO_2} & + & \chemform{H_2} \\\hline
		$\dot{n}_i$ & $\dotc{n}{CO}$ && $\dotc{n}{H_2O}$ && $\dotc{n}{CO_2}$  && $\dotc{n}{H_2}$  \\
		$\dot{n}_f$	& 0 && $\dotc{n}{H_2O}-\dotc{n}{CO}$ && $\dotc{n}{CO_2}-\dotc{n}{CO}$ && $\dotc{n}{H_2}-\dotc{n}{CO}$ \\
	\end{tabular}
	\caption{\label{tab:wgs}Avancement de la réaction Water Gas Shift}
\end{table}


\subsection{Condensation et absorption}
Durant cette étapes aucune réaction chimique n'a lieu. Cette étape permet de retirer tout l'\chemform{H_2O} et le \chemform{CO_2} pour la suite du procédé. On considère que cette étape s'effectue complètement et qu'il ne reste plus aucune trace d'\chemform{H_2O} ou de \chemform{CO_2}.

\subsection{Synthèse de l'ammoniac}
Durant la dernière étape du procédé, nous synthétisons de l'ammoniac à partir d'hydrogène et d'azote :
\begin{chemeqn} 3H_2 + N_2 \longrightarrow 2NH_3 \end{chemeqn}

\begin{table}[h]
	\centering\renewcommand{\arraystretch}{1.2}
	\begin{tabular}{l|ccccc}
		& \chemform{3H_2} & + & \chemform{N_2} & $\longrightarrow$ & \chemform{2NH_3}\\\hline
		$\dot{n}_i$ & $\dotc{n}{H_2}$ && $\dotc{n}{N_2}$ && 0 \\
		$\dot{n}_f$	& $\dotc{n}{H_2}-3\dotc{n}{N_2}$ && 0  && $2\dotc{n}{N_2}$\\
	\end{tabular}
	\caption{\label{tab:synthese}Avancement de la synthèse de l'ammoniac}
\end{table}

\subsection{Air séparation unit}
Maintenant que nous connaissons les quantités nécessaires de \chemform{N_2} ainsi que celles de \chemform{O_2} qui interviennent dans l'ATR, nous pouvons déterminer la quantité d'air entrant dans l'unité de séparation d'air. Dans cette unité rentre de l'air composé de \chemform{O_2}, \chemform{N_2} et \chemform{H_2O}. En pourcentage cela nous donne : 21\% de \chemform{N_2} et 79\% de \chemform{O_2}. L'eau contenue dans l'air est négligeable et ressort directement après
séparation.
Le débit d'air entrant ainsi que les excès seront calculés numériquement dans la section suivante.

\section{Calcul numérique des débits}
Nos paramètres ainsi que leurs valeurs pour le cas précis sont les suivants :
\begin{table}[h]
	\centering\renewcommand{\arraystretch}{1.1}
	\begin{tabular}{lccl}\hline
		Paramètre & Valeur & Unité & Description \\\hline
		$\dotc{m}{CH_4}$ & 800 & \si{\tonne\per\jour} & Débit massique d'alimentation de \chemform{CH_4} \\
		$\nicefrac{\text{\chemform{O_2}}}{\text{\chemform{CH_4}}}$ & 0.6 & - & Rapport $\nicefrac{\text{\chemform{O_2}}}{\text{\chemform{CH_4}}}$ à l'entrée de l'ATR \\
		$\nicefrac{\text{\chemform{H_2O}}}{\text{\chemform{CH_4}}}$& 1.5 & - & Rapport $\nicefrac{\text{\chemform{H_2O}}}{\text{\chemform{CH_4}}}$ à l'entrée de l'ATR \\
		$T_{\text{ATR}}$ & 1200 & \si{\kelvin} & Température de la zone reforming de l'ATR \\
		$p_{\text{ATR}}$ & 50   & \si{\bar} & Pression d'opération de l'ATR \\\hline
	\end{tabular}
	\caption{\label{tab:parametres}Paramètres influant le fonctionnement du procédé}
\end{table}
\subsection{ATR}
\subsubsection{Zone de combustion}
\begin{table}[h]
	\centering\renewcommand{\arraystretch}{1.2}
	\begin{tabular}{ll|ccccccc}
		&& \chemform{CH_4} & + & \chemform{2O_2} & $\longrightarrow$ & \chemform{CO_2} & + & \chemform{2H_2O} \\\hline
		$\dot{m}_i$ & $[\si{\tonne\per\jour}]$ & 800 && 960 && 0 && 1350 \\
		$\dot{m}_i$ & $[\si{\kilo\gram\per\second}]$ \\
		$\dot{n}_i$ & $[\si{\mega\mol\per\jour}]$ & 50 && 30 && 0  && 75  \\\hline	
		$\dot{m}_f$ & $[\si{\tonne\per\jour}]$ & 560 && 0 && 660 && 1890 \\
		$\dot{m}_f$ & $[\si{\kilo\gram\per\second}]$ \\
		$\dot{n}_f$ & $[\si{\mega\mol\per\jour}]$ & 35 && 0 && 15 && 105 \\
	\end{tabular}
	\caption{\label{tab:rcombustion}Résultat de la combustion}
\end{table}

\subsubsection{Zone de reformage}
Suite à la résolution des équations nous obtenons les valeurs suivantes pour $\xi$ et $\beta$ :
$$\begin{cases}\xi &= \SI{28.452}{\mega\mol\per\jour}\\ \beta &= \SI{1.075}{\mega\mol\per\jour}\end{cases}$$
\begin{table}[h]
	\centering\renewcommand{\arraystretch}{1.2}
	\begin{tabular}{ll|ccccccc}
		&& \chemform{CH_4} & + & \chemform{H_2O} & $\rightleftharpoons$ & \chemform{3H_2} & + & \chemform{CO} \\\hline
		$\dot{m}_i$ & $[\si{\tonne\per\jour}]$ &  &&  &&  &&  \\
		$\dot{m}_i$ & $[\si{\kilo\gram\per\second}]$ \\
		$\dot{n}_i$ & $[\si{\mega\mol\per\jour}]$ & 35 && 105 && 0  && 0  \\\hline	
		$\dot{m}_f$ & $[\si{\tonne\per\jour}]$ &  \\
		$\dot{m}_f$ & $[\si{\kilo\gram\per\second}]$ \\
		$\dot{n}_f$ & $[\si{\mega\mol\per\jour}]$ & 6.548 && 75.473 && 86.431 && 27.377 \\
	\end{tabular}
	\caption{\label{tab:rcombustion}Résultat de la combustion}
\end{table}
\begin{table}[h]
	\centering\renewcommand{\arraystretch}{1.2}
	\begin{tabular}{ll|ccccccc}
		&& \chemform{CO} & + & \chemform{H_2O} & $\rightleftharpoons$ & \chemform{H_2} & + & \chemform{CO_2} \\\hline
		$\dot{m}_i$ & $[\si{\tonne\per\jour}]$ &  &&  &&  &&  \\
		$\dot{m}_i$ & $[\si{\kilo\gram\per\second}]$ \\
		$\dot{n}_i$ & $[\si{\mega\mol\per\jour}]$ & 35 && 105 && 0  && 0  \\\hline	
		$\dot{m}_f$ & $[\si{\tonne\per\jour}]$ &  \\
		$\dot{m}_f$ & $[\si{\kilo\gram\per\second}]$ \\
		$\dot{n}_f$ & $[\si{\mega\mol\per\jour}]$ & 6.548 && 75.473 && 86.431 && 27.377 \\
	\end{tabular}
	\caption{\label{tab:rcombustion}Résultat de la combustion}
\end{table}







\end{document}


