\documentclass[a4paper,french]{article}

\usepackage[T1]{fontenc}
\usepackage[utf8x]{inputenc}
\usepackage{chemist}
\usepackage{siunitx}
\usepackage{lmodern}
\usepackage{babel}
\usepackage{float}
\usepackage{fullpage}
\usepackage{amsmath}


\title{Thématique 2 : Gestion de la production}
\author{Groupe 11.64}
\date{Séance 1 : 23/09/2015\\Dernière mise à jour: \today}

\begin{document}
	\maketitle	
	\section{Synthèse de l'ammoniac}
		\begin{chemmath}
			3H_2 + N_2 \longrightarrow 2NH_3	
		\end{chemmath}
		\begin{table}[h]
			\centering
			\renewcommand{\arraystretch}{2}
			\begin{tabular}{|cl|cl|cl|}\hline
				\chemform{3H_2} & $m = \SI{250e9}{\gram}$ & \chemform{N_2} & $m = \SI{1167e9}{\gram}$ & \chemform{2NH_3} & $m = \SI{1417e9}{\gram}$ \\
				& $n = \SI{125e6}{\mol}$ & & $n = \SI{41.7e6}{\mol}$ & & $n = \SI{83.3e6}{\mol}$ \\
				& $M = \SI{2.0}{\gram\per\mol}$ & & $M = \SI{28}{\gram\per\mol}$ & & $M = \SI{17}{\gram\per\mol}$ \\\hline
			\end{tabular}
		\end{table}
	\section{Aspect thermique}
		\paragraph*{Données}
			\begin{itemize}
				\item Température réacteur : $\SI{500}{\celsius}$
				\item Débit eau : $\SI{630}{\meter^3\per\hour}$
				\item Température eau (amont réacteur) : $\SI{20}{\celsius}$
			\end{itemize}	
		\paragraph*{Calculs}
			Nous utilisons les données de la table II du document "La thermodynamique technique". Cette table nous fourni les chaleurs molaires moyennes à pression constante $|\mu c_p|_0^t$ des principaux gaz en \si{\kilo\joule\per\kilo\mol\kelvin}. La chaleur molaire moyenne dans l'intervalle 0 à $t\si{\celsius}$ y est définie par la relation :
			$$ |\mu c_p|_0^t = \frac{1}{t}\int_0^t\mu c_pdt $$
			Pour chaque molécule de la réaction nous calculons la chaleur molaire de \SIrange{25}{500}{\celsius} à l'aide de la formule : 
			$$ \int_{25}^{500} c_{p,m} = (500\cdot|\mu c_p|_0^{500})-(25\cdot|\mu c_p|_0^{25}) $$
			Nous obtenons donc : 
			\begin{table}[h!]\centering\renewcommand{\arraystretch}{2}
			\begin{tabular}{lll}
				$\int_{25}^{500} c_{p,m,\text{\chemform{NH_3}}}$ &= $(500\cdot42.42)-(25\cdot35.26)$ &= \SI{20238.5}{\joule\per\mol} \\
				$\int_{25}^{500} c_{p,m,\text{\chemform{H_2}}}$ &= $(500\cdot29.25)-(25\cdot28.62)$ &= \SI{13909.5}{\joule\per\mol} \\
				$\int_{25}^{500} c_{p,m,\text{\chemform{N_2}}}$ &= $(500\cdot29.86)-(25\cdot29.12)$ &= \SI{14202.25}{\joule\per\mol} 
			\end{tabular}
			\end{table}
			
			\noindent À ces résultats nous ajoutons les données de la table VII du document cité plus haut. Cette table nous fourni les chaleurs de réaction dans les conditions standard de quelques transformations chimiques importantes : 
			$$C_{p,298}^0 = -\Delta\mu H_{298}^0$$
			Pour chaque molécule nous calculons sa variation d'enthalpie à \SI{500}{\celsius} à l'aide de la formule :
			$$ n\cdot\Delta H(500) = n\cdot\left(-C_{p,298}^0+\int_{25}^{500}c_{p,m}\right) $$
			Nous obtenons donc : 
			\begin{table}[h!]\centering\renewcommand{\arraystretch}{2}
				\begin{tabular}{lll}
					$2\cdot\Delta H_{\text{\chemform{NH_3}}}(500)$ &= $2\cdot(-46200+20238.5)$ &= \SI{-57923}{\joule\per\mol} \\
					$3\cdot\Delta H_{\text{\chemform{H_2}}}(500)$ &= $3\cdot(0+13909.5)$ &= \SI{41728.5}{\joule\per\mol} \\
					$\Delta H_{\text{\chemform{N_2}}}(500)$ &= $(0+14202.25)$ &= \SI{14202.25}{\joule\per\mol} 
				\end{tabular}
			\end{table}

			\noindent Nous pouvons ensuite calculer la variation d'enthalpie totale de la réaction : 
			\begin{align*}
				\Delta H = 2\Delta H_{\text{\chemform{NH_3}}} - 3\Delta H_{\text{\chemform{H_2}}} - \Delta H_{\text{\chemform{N_2}}} &= \SI{-108}{\kilo\joule\per\mol} \text{ pour \chemform{2NH_3}} \\ 
				&= \SI{-54}{\kilo\joule\per\mol} \text{ pour \chemform{NH_3}}
			\end{align*}
			La formation d'une mole de \chemform{NH_3} produit donc \SI{54}{\kilo\joule\per\mol}. En sachant que nous produisons \SI{83.3e6}{\mol} par jour, nous pouvons calculer l'énergie totale dégagée par jour :
			$$ \num{83.3e6}\cdot\num{54} = \SI{4.50e9}{\kilo\joule}\text{ par jour} $$
			Pour refroidir le réacteur, nous avons un échangeur de chaleur avec un débit d'eau à \SI{20}{\celsius} à \SI{630000}{\liter\per\hour}. Cela nous fait donc \SI{15.12e6}{\liter} par jour. En sachant qu'il faut \SI{4.185}{\kilo\joule} pour augmenter de \SI{1}{\celsius} \SI{1}{\liter} d'eau, il nous faudrait pour chauffer toute l'eau d'un degré une énergie totale de : 
		   $$ \num{15.12e6}\cdot\num{4.185} = \SI{63.28e6}{\kilo\joule} $$	
		   Enfin, la variation de température nous est donnée par le rapport entre l'énergie dégagée par la réaction et l'énergie nécessaire pour chauffer un litre d'eau d'un degré :
		   $$ \frac{\num{4.50e9}}{\num{63.28e6}} = \SI{71.11}{\celsius} $$
		   En additionnant cette variation avec la température initiale de l'eau, nous obtenons la température de l'eau à la sortie de l'échangeur :
		   $$ \num{71.11} + \num{20.00} = \SI{91.11}{\celsius} $$
		  
	\section{Réactions ATR I et WGS}
		ATR I :
		\begin{chemmath}
			2O_2 + 2CH_4 + H_2O \longrightarrow CH_4 + 3H_2O + CO_2
		\end{chemmath}
		\indent WGS :
		\begin{chemmath}
			2H_2O + CO_2 + CO + H_2 \longrightarrow H_2O + 2CO_2 + 2H_2
		\end{chemmath}
	\section{Bilan des données}
		\begin{table}[h]
			\begin{tabular}{ll}
				0 & Quelle quantité de \chemform{O_2} et \chemform{N_2} nécessaire? \\
				1a & Quelle quantité d'énergie libérée lors de la combustion? \\
				1a & Quelle quantité de \chemform{CH_4} et \chemform{H_2O} \\
				& (Pourquoi rentrée d'\chemform{H_2O} dans zone de combustion?) \\
				1b & Zone de reformage $\rightarrow$ 2 réactions incomplètes \\
				& $\circ$ $K_c$ des 2 réactions \\
				& $\circ$ Energie libérée ou absorbée par chaque réaction \\
			2 & Energie absorbée ou libérée par la réaction \\
			3 & Technique d'élimination du \chemform{CO_2}/\chemform{H_2O} + énergie \\
			4 & Voir séance 1 \\
			& Température dans chaque bloc
			\end{tabular}
		\end{table}
\end{document}
