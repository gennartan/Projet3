\documentclass[a4paper]{report}
\usepackage[T1]{fontenc}
\usepackage[utf8x]{inputenc}
\usepackage[francais]{babel}

\usepackage{amsmath}
\usepackage{multicol}
\usepackage{multirow}
\usepackage{fullpage}
\usepackage{array}
\usepackage{tabularx}

\begin{document}


\begin{center}
\begin{footnotesize}
\begin{tabular}{|c|c|l|l|}
    \hline
    \multicolumn{2}{|c|}{\multirow{2}{*}{\normalsize \textbf{Groupe 12.64}}} & \multirow{4}{8cm}{\centering\large \textbf{Cahier des charges d'une ligne de transfert}} & Date: 07/10/2015 \\ \cline{4-4} 
    \multicolumn{2}{|l|}{} & & Version 1 \\ \cline{1-2} \cline{4-4} 
    \multicolumn{2}{|c|}{Mise à jour} & & Remplace \\ \cline{1-2} \cline{4-4} 
    \textbf{Date} & \textbf{Origine} & & Version 1 07/10/2014 \\ \hline
    \multicolumn{4}{|l|}{\textbf{Contexte}:} \\ 
    \multicolumn{4}{|p{15cm}|}{Concevoir et dimensionner une ligne de transfert d'acide chlorhydrique depuis son unité de fabrication jusqu'à un tank de stockage.} \\ \hline
     & & \multicolumn{2}{l|}{\textbf{Fonctions principales}} \\
    07/10/2015 & Groupe & \multicolumn{2}{p{11cm}|}{FP1. Transporter de l'acide chlorhydrique de la zone de fabrication à la zone de stockage en toute sécurité} \\
    \hline
     & & \multicolumn{2}{p{11cm}|}{\textbf{Critères et niveaux des FP}} \\ \hline
     & & \multicolumn{2}{p{11cm}|}{\textbf{Fonctions de contraintes}} \\
    07/10/2015 & Groupe & \multicolumn{2}{p{11cm}|}{FC1. Equipement imposé} \\
    07/10/2015 & Groupe & \multicolumn{2}{p{11cm}|}{FC2. Conditions imposées (température, pression, distance, niveau, débit)} \\
    07/10/2015 & Groupe & \multicolumn{2}{p{11cm}|}{FC3. Sécurité}\\ \hline
     & & \multicolumn{2}{p{11cm}|}{\textbf{Critères de contraintes}} \\
    07/10/2015 & Groupe & \multicolumn{2}{p{11cm}|}{CC1.1. 4 angles droits} \\
    07/10/2015 & Groupe & \multicolumn{2}{p{11cm}|}{CC1.2. Hauteur du tank=12m} \\
    07/10/2015 & Groupe & \multicolumn{2}{p{11cm}|}{CC1.3. Niveau liquide acceptable=8m} \\
  07/10/2015 & Groupe & \multicolumn{2}{p{11cm}|}{CC1.4. Débit max de la pompe 2T/h} \\
    07/10/2015 & Groupe & \multicolumn{2}{p{11cm}|}{CC1.5. Débit max du compresseur 3T/h} \\
    07/10/2015 & Groupe & \multicolumn{2}{p{11cm}|}{CC2.1. Pression initiale=5barg } \\
    07/10/2015 & Groupe & \multicolumn{2}{p{11cm}|}{CC2.2. Distance=300m} \\
    07/10/2015 & Groupe & \multicolumn{2}{p{11cm}|}{CC2.3. Débit 1T/h} \\
    07/10/2015 & Groupe & \multicolumn{2}{p{11cm}|}{CC2.4. Pression finale=10barg} \\
    07/10/2015 & Groupe & \multicolumn{2}{p{11cm}|}{CC2.5. Pression de design=15barg} \\
    07/10/2015 & Groupe & \multicolumn{2}{p{11cm}|}{CC2.6. Pression de décharge maximum de la pompe et du compresseur=20barg} \\
    07/10/2015 & Groupe & \multicolumn{2}{p{11cm}|}{CC2.7. Température de l'acide=200°C} \\
    07/10/2015 & Groupe & \multicolumn{2}{p{11cm}|}{CC3.1. Contrôle automatique du transfert} \\
    07/10/2015 & Groupe & \multicolumn{2}{p{11cm}|}{CC3.2. Barrière de sécurité supplémentaire} \\ \hline
\end{tabular}
\end{footnotesize}
\end{center}
\end{document}
