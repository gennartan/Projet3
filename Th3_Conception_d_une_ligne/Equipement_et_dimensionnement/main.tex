\documentclass[french, a4paper, 10pt]{article}

\usepackage{babel}
\usepackage[T1]{fontenc}
\usepackage[utf8x]{inputenc}

\usepackage{fullpage}
\usepackage{float}
\usepackage{nicefrac}

\usepackage{chemist}
\usepackage{siunitx}
\usepackage{amsmath}

\DeclareSIUnit\jour{j}
\sisetup{per-mode=fraction, fraction-function=\nicefrac}
\newcommand{\dotc}[2]{\dot{#1}_{\text{\chemform{#2}}}}

\title{Thématique 3 : Equipement et dimensionnement d'une ligne de transfert}
\author{Groupe 12.64}
\date{\today}

\begin{document}
\maketitle
\section{Contraintes de la ligne de transfert}
\begin{itemize}
\item $Pression_{initiale}=5bar$
\item $\Delta X=300m$
\item 4 coudes de 90°
\item Débit 1T/h
\item $Pression_{finale}=10bar$
\end{itemize}

\section{Caractéristiques de l'équipement existant}
\begin{itemize}
\item Tank hauteur=12m
\item niveau liquide acceptable=8m
\item pression de design=15bar
\item pression de décharge maximum de la pompe et du compresseur 20 bar
\item débit max de la pompe 2t/h
\item débit max du compresseur 3t/h
\end{itemize}

\section{Indications}
Quelques indications pour nous aider à la résolution de notre cahier des charges
\begin{itemize}
\item perte de pression lié à une vanne de contrôle=2bar
\item perte de pression lié à la présece d'u coude (90°):équivalent à la perte de charge généré par 20*D mètres le long de la ligne de transfert droite, où D est le diamètre de la ligne
\item Darcy-Weisbach $f_{D}=0.02$
\end{itemize}

\section{Cahier des charges}
\subsection{Contrôle automatique du transfert}
Evite le sur-remplissage du tank
\subsection{Barrière de sécurité supplémentaire}
En cas de défaillance du système
\subsection{Schéma de principe de la ligne de transfert}

\section{Dimensions de la conduite}
Assurer le transfert et la pression

\section{Mini analyse de risques}
Grâce à HAZOP sur laquelle nous aurons plus d'information le 18/11

\end{document}
